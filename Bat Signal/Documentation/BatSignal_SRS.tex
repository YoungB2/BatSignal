\documentclass[10pt,a4paper]{article}

\author{Bryan Young}
\title{BatSignal:\\Software Requirements Specification}
\date{1 June 2015}

\begin{document}
\maketitle
\newpage

\tableofcontents{}
\newpage

\section{Introduction}
\subsection{Purpose}
%Full description of the main objectives of the SRS.
\textnormal{The Software Requirements Specification for the BatSignal system describes in detail the software components running on and operating the distributed sensor network. This specification will elaborate the necessary performance thresholds and functionality requirements the BatSignal distributed sensor network is expected to meet.}

\subsection{Scope of Development}
%Identifies the product to be developed by name and function, lists limitations(if any), highlights distinct features, lists benefits as clearly and precisely as possible. This will provide the basis for the brief description of your product.
\textnormal{
\begin{enumerate}
	\item{Sensor Driver}
	\begin{enumerate}
		\item{Record hardware sensor data}
		\item{Communicate on sensor node network}
	\end{enumerate}
	\item{Command and Control Driver}
	\begin{enumerate}
		\item{Listen on the sensor node network}
		\item{Communicate with Google APIs (Application Programming Interface)}
		\item{Alert administrators when certain conditions are met}
	\end{enumerate}
\end{enumerate}
}

\subsection{Definitions, Acronyms, and Abbreviations}
%Include any specialized terminology dictated by the application area or the product area. This will help the reader understand the rest of the text. Be sure to alphabetize! If this section becomes longer than one page, move the definitions, etc. to an Appendix and provide a pointer in this section.
\subsubsection{Application Specific Definitions}
\textnormal{
\begin{center}
\begin{tabular}{ |l|l| }
	% Defines acronyms, abbreviations, or names pertaining specifically to the BatSignal software.
	\hline
	\multicolumn{2}{|c|}{Application Specific Definitions} \\
	\hline
	% Acronym & Definition \\
	% Order alphabetically
	ACK & Definition \\
	\hline
\end{tabular}
\end{center}
}

\subsubsection{Industry Definitions}
\textnormal{
\begin{center}
\begin{tabular}{ |l|l| }
	% Defines acronyms, abbreviations, or names pertaining third-party devices, applications, or organizations.
	\hline
	\multicolumn{2}{|c|}{Industry Definitions} \\
	\hline
	% Acronym & Definition \\
	% Order alphabetically
	B.A.T.M.A.N & Better Aproach to Mobile Adhoc Networking \\
	RPi & Raspberry Pi \\
	WiPi & Wireless network adapter for the Raspberry Pi \\
	WIT & Wentworth Institute of Technology \\	
	\hline
\end{tabular}
\end{center}
}

\subsubsection{Technical Definitions}
\textnormal{
\begin{center}
\begin{tabular}{ |l|l| }
	% Defines acronyms, abbreviations, or names pertaining to technologies, tools, and other non-application or industry terms.
	\hline
	\multicolumn{2}{|c|}{Technical Definitions} \\
	\hline
	% Acronym & Definition \\
	% Order alphabetically
	API & Application Programming Interface \\
	\hline
\end{tabular}
\end{center}
}

\subsection{References}
\textit{Mention books, articles, web sites, worksheets, people who are sources of information about the application domain, etc.
\newline\newline Use proper and complete reference notation. If this section becomes longer than one page, move the references to an Appendix and provide a pointer in this section.}

\subsection{Overview}
% A short description of how the rest of the SRS is organized and what can be found in the rest of the document.
\textnormal{The following sections will describe the general aspects of the project before listing the specific software requirements. }

\section{General Description}
\subsection{User Characteristics}
%This section considers the needs of the anticipated users. List critical characteristics of the system's human interfaces based on the anticipated users' characteristics.
\textnormal{The BatSignal distributed sensor network interacts with both administrative medical staff and elderly patients. The system will send email alerts to a designated set of medical staff when it detects emergency events. The system will passively interact with elderly patients by recording audio captures and will require no direct interaction.}

\subsection{Product Perspective}
% If the product is stand-alone, give the relationship to other products.
% If the product is part of a larger product, then identify its interface to the other products.
% Identify the product's external interfaces with its environment.
% If the product uses existing hardware, describe the hardware.
% If the product requires new hardware, give a detailed explanation of the hardware.
\textnormal{The BatSignal distributed sensor network is broken down into two primary components: the controller node and the sensor node. \\\\
The controller is responsible for communicating with third-party APIs, processing the text of converted audio captures, and dispatching SMTP alerts. \\\\
The sensor node responsible for capturing audio input and sending captures to the controller.}


\subsection{Overview of Functional Requirements}
\textit{Provide a short description of the functions to be performed by the software, i.e. what the product should do. This description must be in a form understandable to users, operators, and clients. The detailed requirements specifications are left to Section 3.2 in this document. If you number the Functional Requirements in a systematic manner, it will be easier for you to refer to them in Section 3.2 of the SRS, in the design document you will write later, and in the testing document (also to be written later). This should not be design-oriented, a common mistake.}

\subsection{Overview of Data Requirements}
\textit{Describe data that are input or output from the product as well as any data that are stored within the system, for example in files or on disc. This section should only cover data requirements from the user's point of view. \newline\newline
Once again, this should not be design-oriented.}

\subsection{General Constraints, Assumptions, Dependencies, Guidelines}
\textit{Include factors that impose constraints on the implementation of the software product. This can include hardware limitations or requirements, the amount of memory available, response times, policies, interfaces to other application software, networks, environmental limitations, compliance with relevant standards. This section can also provide guidance in situations when there may be more than one implementation strategy.}

\subsection{User View of Product Use}
\textit{This section will provide a user's-eye-view of the product.
This may include aspects such as narrative to describe the setting, sketches to show possible appearance of the screen, samples of the data that is stored, entered, or output, and dramatic scenarios that demonstrate the product in operation. If this section becomes longer than about two pages, then break some parts into Appendices and provide pointers from within the text of this section.
}

\section{Specific Requirements}
\subsection{External Interface Requirements}
\textit{
\begin{itemize}
	\item{operator/user interface characteristics from the human factors point of view}
	\item{characteristics required of the interface between the software product and each of the hardware components}
	\item{interfaces with other software components or products, including other systems, utility software, databases, and operating systems}
\end{itemize}
}

\subsection{Functional Requirements}
\textit{A detailed list of functional requirements and their descriptions.}

\subsubsection{Functional Requirement Template}
\textit{This lists the exact template your SRS will apply in describing each of the functional components that were identified in Section 2.3. You may also look at the DFD to help you to identify your functional components. For EACH functional component, you should have a section. Each of these sections should be at least the following:
\begin{itemize}
	\item{purpose / description}
	\item{inputs: which inputs; in what form/format will inputs arrive; from what sources input will be derived, legal domains of each input element}
	\item{processing: describes the *outcome* rather than the *implementation*; include any validity checks on the data, exact timing of each operation (if needed), how to handle unexpected or abnormal situations}
	\item{outputs: the form, shape, destination, and volume of the output; output timing; range of parameters in the output; unit measure of the output; process by which the output is stored or destroyed; process for handling error messages produced as output}
\end{itemize}
}

\subsection{Non-Functional Requirements}
\subsubsection{Performance}
\begin{enumerate}
	\item{\textbf{BatSignal Alert Dispatch Throughput Time}} \\
	The BatSignal distributed sensor network should have a throughput time of no more than 5000 $\mu$s. This is the time from audio capture to alert dispatch in the case of a positive hit for an emergency keyword.
	\item{\textbf{Sensitivity (True Positive) and Specificity (True Negative)}} \\
	Both the sensitivity and specificity of the system should be above 90\% accuracy.
\end{enumerate}

\subsubsection{Reliability}

\subsubsection{Availability}

\subsubsection{Security}

\subsubsection{Maintainability}

\subsubsection{Portability}

\subsection{Quality Attributes}

\section{Other Requirements}

\section {Use Cases}

\section{Project Planning and Risk Management}

\appendix
\section{Appendix}

\end{document}